% !TEX root =  ../main.tex
\section{Writing Process}

Writing is a very iterative process and over time the written text gets \emph{massaged} into its final form.
To make this process easy, it is recommended to keep as much information as possible in the document and not use external documents for planning, communication, or extra notes.

The paper will be very messy initially, but the paper should make it trivial to get an estimate of the current state \emph{on a quick glance}.
As such, we will use various Latex macros to "label" the text.

The most basic indicator is \cmd{todo}.
Use it to note down ideas during a discussion or when you do not have the time to elaborate something.
\todo{Extend this explanation}

Once you had time to add more information, the text is likely still in a rough shape.
Use \cmd{rev} to mark text that is \emph{in principal} already \emph{content complete}, but that still requires a thorough revision.
\rev{This makes it a lot easier to collaborators to understand which parts are still rough or which parts should be considered "done".

Do not be shy and remove the \cmd{rev} tag as soon as you think that the text is polished.
Should a collaborator feel that more work is need in a section, they will add the marker back.}

Scientific paper usually contains many numbers.
To make it easy to spot the numbers when you check your paper for consistency before submission, wrap them in a \cmd{checkNum} command.
You will find it easier to spot the \checkNum{five} mentions of the \cmd{cmd} command in your paper when the number is highlighted.

All these commands are predefined in the COPS-flavored document class that extends the basic IEEE template.
You want to make sure that the formatting is removed before you submit the paper.
You can find commented out \cmd{RenewCommand} instructions in the \code{preface.tex}, once you uncomment them, all formatting and extra text is gone.
Try it out!

Finally, do not include questions or instructions as comments, as someone who will only read the PDF won't see them.
Also, trust your version control system and avoid keeping commented text in your paper.

