% !TEX root =  ../main.tex
\section{Citations \& References}

Citing papers or external resources is very easy in Latex.
In our field, the convention is to use the first authors \emph{last name} and \emph{et al.} to indicate that there were more authors. For example: Smith et al.~\cite{AAI34-icse} were the first to introduce foobar by perfoming blah, which lead to blubb.
Please note the {\textasciitilde} in the Latex code, which makes sure that the citation number will never break into the next line alone, which would look weird.
This style of citation also works for all other venues like journals~\cite{DE23-emse} or books~\cite{ABC12-abc}.
It is also possible to cite websites~\cite{Pue08}, however, these pages should then contain relevant content for the paper.
When the reference is only linking to a tool, it is often better to use a footnote for that.%
\footnote{\url{https://www.maven.org}, accessed: 10-Nov-2023}

As an important note, make sure that you \emph{never} use a citation as an object in a sentence (like "As shown in~\cite{DE23-emse}, ..."), this is very bad style.
Instead, either name the authors (e.g., "Foo et al.~\cite{DE23-emse} have shown ...") or formulate it as a general statement (e.g., "Prior research has shown that ... \cite{DE23-emse}").

\medskip
Within a document, it is possible to refer to each element that has a counter, like sections or figures.
When a \cmd{label} has been added to these elements, one can use \cmd{ref} to refer back to them.
This way, it is possible to refer back to Section~\ref{sec:intro} or Figure~\ref{fig:abstract-image}.
Please note that both Section and Figure are uppercase and that the Latex code again uses a {\textasciitilde} to prevent unfortunate linebreaks.
Please also not that the prefixes \code{fig:} and \code{sec:} have been added \emph{by convention} to the \cmd{label} and \cmd{ref}.

To make it easier to reference elements in the document, the template supports the \code{cleverref} package.
It is possible to just use \cmd{Cref} to refer to the elements, without explicitly typing the elements name.
For example, note in the Latex source how the following reference to \Cref{fig:abstract-image} does not contain the "Figure", which appears in the .pdf though.

